\documentclass{article}

% --- packages ---

\usepackage[utf8]{inputenc}
\usepackage[english, ngerman]{babel}
\usepackage{amsmath}
\usepackage{amsthm}
\usepackage{tikz, pgfplots}
\usepackage{amssymb}
\usepackage{graphicx}
%\usepackage{subcaption}
\usepackage{pgf}
\usepackage{caption}
\usepackage{listings, lstautogobble}
\usepackage{delarray}
\usepackage{bigints}
\usepackage{float}
\usepackage{layouts}
\usepackage{subfigure}
\usepackage{url}
\usepackage{fancyhdr}
\usepackage{MnSymbol}
\usepackage{wasysym}
\usepackage[a]{esvect} %für bessere Vektorpfeile
\usepackage{setspace}
\usepackage{geometry}
\usepackage{hyperref}
\usepackage{acronym}
\usepackage{ifsym}
\usepackage[sorting=nty]{biblatex}
\usepackage[nottoc,numbib]{tocbibind}

% --- Input macros von Ide ---

%\input{macros}

% --- Seitenränder / Einrücktiefe ---

\geometry{left=2.5cm, right=2.5cm, top=2cm, bottom=2cm}
\parindent0cm

% --- new commands ---

\newcommand{\RM}[1]{\MakeUppercase{\romannumeral #1{}}} %für römische Zahlen #1{}}} %für römische Zahlen
\newcommand{\xvec}{\underline{x}}
\newcommand{\bvec}{\underline{b}}
\newcommand{\uvec}{\underline{u}}
\newcommand{\vvec}{\underline{v}}
\newcommand{\wvec}{\underline{w}}
\newcommand{\pvec}{\underline{p}}
\newcommand{\zvec}{\underline{z}}
\newcommand{\xstern}{\underline{x}^*}
\newcommand{\Rmn}{\mathbb{R}^{m \times n}}
\newcommand{\Elasticsearch}{\textit{Elasticsearch}\;}
\newcommand{\Rnn}{\mathbb{R}^{n \times n}}
\newcommand{\rang}{\text{rang}}
\def\code#1{\texttt{#1}}
\newcommand{\dom}{\textrm{dom}}
\newcommand{\lojoin}{{\tiny \textifsym{d|><|}}}
\newcommand{\rojoin}{{\tiny \textifsym{|><|d}}}
\newcommand{\fojoin}{{\tiny \textifsym{d|><|d}}}
\newcommand{\m}{\cdot} %Malzeichen einfacher
\newcommand{\entspricht}{$\mathop{\hat{=}}$} %Entspricht-Zeichen
\renewcommand{\labelnamepunct}{\addcolon\space}

% -- Definitionen für Theoremumgebungen ---

\newtheoremstyle{newline}% name
{}% Space above
{\baselineskip}% Space below
{\normalfont}% Body font
{}% Indent amount
{\bfseries}% Theorem head font
{:}% Punctuation after theorem head
{\newline}% Space after theorem head
{\thmname{#1}\thmnumber{ #2}\thmnote{ (#3)}}% Theorem head spec (can be left empty, meaning ‘normal’ )

\theoremstyle{newline}
\newtheorem{definition}{Definition}[section]
\newtheorem{satz}[definition]{Satz}
\newtheorem{bemerkung}[definition]{Bemerkung}
\newtheorem{folgerung}[definition]{Folgerung}
\renewenvironment{proof}[1][\proofname:]{%
\minisec{#1}
\pushQED{\qed}%
\itshape
}{%
\popQED
\par
\medskip
}

% --- listing_settings ---


\definecolor{red}{rgb}{0.6,0,0} % for strings
\definecolor{blue}{rgb}{0,0,0.6}
\definecolor{green}{rgb}{0,0.8,0}
\definecolor{cyan}{rgb}{0.0,0.6,0.6}

\lstset{autogobble=true}

% --- Literaturverzeichnis ---


% --- Beginn des eigentlichen Dokuments ---


\begin{document}

% -- Layout Kopfzeile ---

\pagestyle{fancy}
\lhead{\small{\slshape}}
\chead{}	
\rhead{\mdseries \leftmark}


\begin{titlepage}

%\includegraphics[width=0.20\textwidth]{Bilder/fh_logo.png}\hfill\includegraphics[width=0.25\textwidth]{Bilder/diamant-software-logo-teams.png}

% --- include correct images here

\centering

{\bfseries HSBI Bielefeld \par}
\vspace{0.25cm}
{\bfseries University of Applied Sciences \par}
\vspace{0.25cm}
{\bfseries Fachbereich Ingenieurwissenschaften und Mathematik \par}
\vspace{0.25cm}
{\bfseries Studiengang Optimierung und Simulation \par}
\vspace{1.5cm}
{\huge\bfseries Lösen von nichtlinearen Gleichungssystemen mit einem Reinforcement-Learning-Agent \par}
\vspace{2.5cm}
{\Large\bfseries Bericht \par}
{\vspace{8.5cm}}

\flushleft
\begin{tabular}{ll}
	Vorgelegt von: & Nicolas Schneider \vspace{0.25cm}\\
	Matrikelnummer: & 1208960 \vspace{0.25cm} \\
	Studiengang: & Optimierung und Simulation \vspace{0.5cm} \\
	Abgabedatum: & 07.04.2024 \vspace{0.5cm} \\
	Betreuer: & Prof. Dr. rer. nat. Bernhard Bachmann
\end{tabular}
\vfill
\end{titlepage}


\thispagestyle{empty}
\newpage



\begin{onehalfspace}

\thispagestyle{empty}
\selectlanguage{english}
\begin{abstract}
	In Folge des digitalen Informationszeitalters, in dem die Anzahl der verfügbaren Informationen exponentiell gestiegen ist, kam der Begriff \textit{Big Data} immer öfters auf. Im Zuge der immer komplexer und größere werdenden Daten, wuchs auch die Popularität der \textit{NoSQL}-Datenbanken, die im Vergleich zu relationalen Datenbanken besonders gut mit großen Datenmengen umgehen können.
	\\
	
	Diese Arbeit beschäftigt sich mit dem Vergleich einer relationalen Datenbank und einer NoSQL-Datenbank, im speziellen die dokumentenorientierte Datenbank \textit{Elasticsearch}, im Bezug auf die Performance, was Suche und Indexierung angeht. Unterdessen wird neben dem funktionalen Aufbau auch die Anwendbarkeit der jeweiligen Datenbank auf die Auswertung von verschiedenen Log-Dateien, wie IIS- und Eventlogs im Unternehmen der Diamant Software GmbH eingegangen. 
\end{abstract}
\selectlanguage{ngerman}
\newpage


\tableofcontents	% Inhaltsverzeichnis		
\thispagestyle{empty}

\newpage		

\section{Einführung}

\section{Grundlagen nichtlinearer Gleichungssysteme}

\section{Grundlagen Reinforcement Learning}

\section{Anwendung eines RL-Agents auf nichtlineare Gleichungssysteme}

\section{Umsetzung}

\section{Ergebnisse}

\section{Fazit}



\end{onehalfspace}

\end{document}








